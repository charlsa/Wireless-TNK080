%Om filen typsätts som del av hela rapporten så finns \master definierat i början och ingen \begin{document} och \end{document} får finnas, men för att kunna typsätta filen för sig är dem ett måste! \newcommand{\master}{} krävs i början på huvudrapporten!
\ifdefined\master
\else
	\documentclass[twocolumn]{article}
	%\input{../preamble}
	\usepackage{graphicx}
	\usepackage{float}

	\begin{document}
\fi

In the first simulation, the users with the lowest throughput appeared at the cell edge. Due to the random fading, some users farther away from the BS got higher throughput and vice versa. Figure \ref{fig:task1} shows the result of one simulation with 100 users. The five users with the lowest throughput are marked with circles.

\begin{figure}[H]
\centering
\fbox{\includegraphics[trim=150 335 185 335, clip=true, width=8cm]{Figure/fig1.pdf}}
\caption{Single BS simulation.}
\label{fig:task1}
\end{figure}

Introducing the second BS, the users with the lowest throughput were instead the ones near the boundary between the cells. In figure \ref{fig:task2} the same users and fading is used, but with the power from the second cell added as noise. 

\begin{figure}[H]
\centering
\fbox{\includegraphics[trim=140 335 155 330, clip=true, width=8cm]{Figure/fig2.pdf}}
\caption{Simulation with one neighbouring cell.}
\label{fig:task2}
\end{figure}

In each of the two simulations the average throughput per user was calculated. The difference between the two cases is quite large, as presented in table \ref{tab:tp}. 

\begin{table}[H]
\centering
\caption{Average throughput results.}
\label{tab:tp}
\begin{tabular}{r r}
Simulation & Throughput \\ \hline
Task 1 & 25.771 Mbps \\
Task 2 & 7.262 Mbps
\end{tabular}
\end{table}


\ifdefined\master
\else
	\end{document}
\fi