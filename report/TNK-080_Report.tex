%bra paket
\documentclass[twocolumn]{article}
\usepackage[utf8]{inputenc}
%\usepackage[swedish]{babel}
\usepackage{fancyhdr}
%\usepackage{times}
%\usepackage{alltt} %verbatim text med möjlighet till andra latexkommandon i.
%\usepackage[usenames,dvipsnames]{color} %fler färger att välja på
%\usepackage{wrapfig} %figurer som ligger sida vid sida med texten
%\usepackage[table]{xcolor} %bakgrundsfärg i tabeller
%\usepackage[small,compact]{titlesec} %Spara plats!!
\usepackage{amsmath}
\usepackage{multicol}
\usepackage{graphicx}
\usepackage{float} %gör så att man kan placera bilder exakt mha [H]
%\usepackage[table]{xcolor} %bakgrundsfärg i tabeller

%\usepackage[ddmmyyyy]{datetime}

%\usepackage{setspace}
\usepackage[usenames,dvipsnames]{color} %fler färger att välja på
%\usepackage{pdfpages} %för att kunna använda includepdf i appendix

% Different font in captions
\newcommand{\captionfonts}{\em}
\makeatletter  % Allow the use of @ in command names
\long\def\@makecaption#1#2{%
  \vskip\abovecaptionskip
  \sbox\@tempboxa{{\captionfonts #1: #2}}%
  \ifdim \wd\@tempboxa >\hsize
    {\captionfonts #1: #2\par}
  \else
    \hbox to\hsize{\hfil\box\@tempboxa\hfil}%
  \fi
  \vskip\belowcaptionskip}
\makeatother   % Cancel the effect of \makeatletter


%c++:
\usepackage{listings}
\definecolor{light-gray}{gray}{0.90}
\lstset{
language=MATLAB,                % choose the language of the code
basicstyle=\ttfamily,       % the size of the fonts that are used for the code
keywordstyle=\color{DarkOrchid},
stringstyle=\color{blue},
commentstyle=\color{OliveGreen},
numbers=left,                   % where to put the line-numbers
numberstyle=\footnotesize,      % the size of the fonts that are used for the line-numbers
%stepnumber=2,                   % the step between two line-numbers. If it's 1 each line 
                                % will be numbered
numbersep=5pt,                  % how far the line-numbers are from the code
backgroundcolor=\color{light-gray},  % choose the background color. You must add \usepackage{color}
%showspaces=false,               % show spaces adding particular underscores
showstringspaces=false,         % underline spaces within strings
%showtabs=false,                 % show tabs within strings adding particular underscores
frame=single,	                % adds a frame around the code
tabsize=1,	                % sets default tabsize
%captionpos=b,                   % sets the caption-position to bottom
breaklines=true,                % sets automatic line breaking
%breakatwhitespace=false,        % sets if automatic breaks should only happen at whitespace
%title=\lstname,                 % show the filename of files included with \lstinputlisting;
                                % also try caption instead of title
%escapeinside={\%*}{*)},         % if you want to add a comment within your code
%morekeywords={*,...}            % if you want to add more keywords to the set
%extendedchars=false
xleftmargin=15pt,
framexleftmargin=0pt,
framexrightmargin=5pt
}


%marginaler
\setlength\topmargin{0in}
\setlength\headheight{11pt}
\setlength\textheight{8.1in}
\setlength\textwidth{6.5in}
\setlength\oddsidemargin{0in}
\setlength\evensidemargin{0in}
\setlength\parindent{0in}
\setlength\parskip{0in}
\frenchspacing %Oui!

%För att kunna typsätta delar för sig!
\newcommand{\master}{}

%då kör vi


\begin{document}
%%%%%%%%%%%%%%%%%%% Försättsblad %%%%%%%%%%%%%%%%%%%%%%%%
\begin{titlepage}
\title{\textbf{Understanding basics of} \\
\textbf{OFDMA \& Cell edges}\\
\Large{Wireless Communication Systems}\\
\large{TNK080}}
\author{
\vspace{30pt}\\
\large
ED3:\bigskip \\
\begin{tabular}{l l}
	Dan	Helgesson & danhe046 \\
	Albert Skog	& albsk635 \\
	Karl Westerberg	& karwe772 \\
	Aron Grundberg	& arogr832\\
\end{tabular}\vspace{40pt}\\
Examiner: Vangelis Angelakis 
}
\date{Submitted: \today}
\maketitle
\thispagestyle{empty}
\begin{center}


\begin{figure}[b]
	\begin{center}
		\includegraphics[scale=0.6]{Figure/LIU-logo.jpg}
	\end{center}
\end{figure}

\end{center}

\end{titlepage}
\clearpage \thispagestyle{empty} ~\clearpage %baksida av försättsblad

%%%%%%%%%%%%%%%%%%% Header %%%%%%%%%%%%%%%%%%%%%%%
\pagestyle{fancy}
\fancyhead[l]{Understanding basics of OFDMA \& Cell edges}
\fancyhead[r]{Dan Helgesson, Albert Skog, Karl Westerberg}
\fancyfoot[c]{}

%%%%%%%%%%%%%%%%%% Abstract %%%%%%%%%%%%%%%%%%%%%%
\onecolumn
\begin{abstract}
asd
\end{abstract}
\clearpage
%%%%%%%%%%%%%%%%%%% Contents %%%%%%%%%%%%%%%%%%%%%

\tableofcontents
\clearpage
\setcounter{page}{1}
\fancyfoot[c]{\thepage}
\twocolumn


%%%%%%%%%%%%%%%%%% Rapporten %%%%%%%%%%%%%%%%%%%%%
\section{Introduction}
Orthogonal Frequency-Division Multiple Access (OFDMA) is a multi-user version of the digital modulation scheme Orthogonal Frequency-Division Multiplexing (OFDM) used in several 4G cell networks and mobile broadband standards. The idea is to transform a high data rate stream into a set of low data rate streams that are transmitted in parallel. The following report demonstrates a simple example of a throughput simulation using OFDMA with round-robin protocol,%REFERENS
 meaning that all parallel data streams (sub-carriers) are dedicated to one user at the time. The problem was visualized simulating a single cell (one Base Station) of 1000m radius and 100 randomly placed users, and for two cells taking interference into account.



\subsection{Purpose}
The purpose was to simulate and comapre the throughput for users served by a single base station in the case of no interference and the case of one interfering base station 2000m away. For each sub-carrier and user the path loss was calculated using frii's free space path loss model, and a rayleigh distributed random variable as the fading factor.


\subsection{Methodology}
The problem was visualized in a MatLab-generated plot and the 5 users with worst throughput was indicated.

\subsection{Hypothesis}
The users with the worst throughput would be expected to be the users farthest out in the case of no interference, with some variance due to the randomness of fading. In the interference case the users closest to the other base station was expected to have the worst throughput.
\section{Method}
Firstly the 100 users was randomly placed within the circle. The distance from origo is needed for the power received calculationand is calculated by the well known Pythagorean theorem. The power received shown in equ. \eqref{power}  is calculated once for each subcarrier. The only thing that varies in this case is the wavelenght for each center frequency.

\begin{equation}
\label{power}
P_r(d)=\frac{P_rG_tG_r\lambda^2}{(4\pi)^2d^2L}
\end{equation}

A fading is added for each subcarrier for each user. The fading is added because nothing is ideal.%mer text?
To calculate the capacity the Shannon formula in \eqref{shannon} was used. The SNR was computed from the $P_r$ and the noise $N$.

\begin{equation}
\label{shannon}
C=BW~ln(1+SNR)
\end{equation}

\section{Results}

%Om filen typsätts som del av hela rapporten så finns \master definierat i början och ingen \begin{document} och \end{document} får finnas, men för att kunna typsätta filen för sig är dem ett måste! \newcommand{\master}{} krävs i början på huvudrapporten!
\ifdefined\master
\else
	\documentclass[twocolumn]{article}
	%\input{../preamble}
	\usepackage{graphicx}
	\begin{document}
\fi

In the first simulation, the users with the lowest throughput appeared at the cell edge. Due to the random fading, some users farther away from the BS got higher throughput and vice versa. Figure \ref{fig:task1} shows the result of one simulation with 100 users. The five users with the lowest throughput are marked with circles.

\begin{figure}[H]
\centering
\fbox{\includegraphics[trim=150 335 185 335, clip=true, width=8cm]{Figure/fig1.pdf}}
\caption{Single BS simulation.}
\label{fig:task1}
\end{figure}

Introducing the second BS, the users with the lowest throughput were instead the ones near 

\begin{figure}[H]
\centering
\fbox{\includegraphics[trim=140 335 155 330, clip=true, width=8cm]{Figure/fig2.pdf}}
\caption{Simulation with one neighbouring cell.}
\label{fig:task1}
\end{figure}

\begin{table}[H]
\centering
\caption{Average throughput results.}
\label{tab:tp}
\begin{tabular}{r l}
Simulation & Throughput \\ \hline
Task 1 & 2.5771 Mbps \\
Task 2 & 0.7262 Mbps
\end{tabular}
\end{table}








\ifdefined\master
\else
	\end{document}
\fi


\section{Conclusion}

\onecolumn
\appendix
\section{Code}
\lstinputlisting{../project2.m}

\end{document}