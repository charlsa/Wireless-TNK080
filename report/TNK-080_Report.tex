%bra paket
\documentclass[twocolumn]{article}
\usepackage[utf8]{inputenc}
%\usepackage[swedish]{babel}
\usepackage{fancyhdr}
%\usepackage{times}
%\usepackage{alltt} %verbatim text med möjlighet till andra latexkommandon i.
%\usepackage[usenames,dvipsnames]{color} %fler färger att välja på
%\usepackage{wrapfig} %figurer som ligger sida vid sida med texten
%\usepackage[table]{xcolor} %bakgrundsfärg i tabeller
%\usepackage[small,compact]{titlesec} %Spara plats!!
\usepackage{amsmath}
\usepackage{multicol}
\usepackage{graphicx}
\usepackage{float} %gör så att man kan placera bilder exakt mha [H]
%\usepackage[table]{xcolor} %bakgrundsfärg i tabeller

%\usepackage[ddmmyyyy]{datetime}

%\usepackage{setspace}
\usepackage[usenames,dvipsnames]{color} %fler färger att välja på
%\usepackage{pdfpages} %för att kunna använda includepdf i appendix

% Different font in captions
\newcommand{\captionfonts}{\em}
\makeatletter  % Allow the use of @ in command names
\long\def\@makecaption#1#2{%
  \vskip\abovecaptionskip
  \sbox\@tempboxa{{\captionfonts #1: #2}}%
  \ifdim \wd\@tempboxa >\hsize
    {\captionfonts #1: #2\par}
  \else
    \hbox to\hsize{\hfil\box\@tempboxa\hfil}%
  \fi
  \vskip\belowcaptionskip}
\makeatother   % Cancel the effect of \makeatletter


%c++:
\usepackage{listings}
\definecolor{light-gray}{gray}{0.85}
\lstset{
language=TeX,                % choose the language of the code
basicstyle=\ttfamily,       % the size of the fonts that are used for the code
keywordstyle=\color{DarkOrchid},
stringstyle=\color{blue},
commentstyle=\color{OliveGreen},
numbers=left,                   % where to put the line-numbers
numberstyle=\footnotesize,      % the size of the fonts that are used for the line-numbers
%stepnumber=2,                   % the step between two line-numbers. If it's 1 each line 
                                % will be numbered
numbersep=5pt,                  % how far the line-numbers are from the code
backgroundcolor=\color{light-gray},  % choose the background color. You must add \usepackage{color}
%showspaces=false,               % show spaces adding particular underscores
showstringspaces=false,         % underline spaces within strings
%showtabs=false,                 % show tabs within strings adding particular underscores
frame=single,	                % adds a frame around the code
tabsize=1,	                % sets default tabsize
%captionpos=b,                   % sets the caption-position to bottom
%breaklines=true,                % sets automatic line breaking
%breakatwhitespace=false,        % sets if automatic breaks should only happen at whitespace
%title=\lstname,                 % show the filename of files included with \lstinputlisting;
                                % also try caption instead of title
%escapeinside={\%*}{*)},         % if you want to add a comment within your code
%morekeywords={*,...}            % if you want to add more keywords to the set
%extendedchars=false
xleftmargin=15pt,
framexleftmargin=0pt,
framexrightmargin=5pt
}


%marginaler
\setlength\topmargin{0in}
\setlength\headheight{11pt}
\setlength\textheight{8.1in}
\setlength\textwidth{6.5in}
\setlength\oddsidemargin{0in}
\setlength\evensidemargin{0in}
\setlength\parindent{0in}
\setlength\parskip{0in}
\frenchspacing %Oui!

%För att kunna typsätta delar för sig!
\newcommand{\master}{}

%då kör vi


\begin{document}
%%%%%%%%%%%%%%%%%%% Försättsblad %%%%%%%%%%%%%%%%%%%%%%%%
\begin{titlepage}
\title{\textbf{Understanding basics of} \\
\textbf{OFDMA \& Cell edges}\\
\Large{Wireless Communication Systems}\\
\large{TNK080}}
\author{
\vspace{30pt}\\
\large
ED3:\bigskip \\
\begin{tabular}{l l}
	Dan	Helgesson & danhe046 \\
	Albert Skog	& albsk635 \\
	Karl Westerberg	& karwe772 \\
	Aron Grundberg	& arogu???\\
\end{tabular}\vspace{40pt}\\
Examiner: Vangelis Angelakis 
}
\date{Submitted: \today}
\maketitle
\thispagestyle{empty}
\begin{center}


\begin{figure}[b]
	\begin{center}
		\includegraphics[scale=0.6]{Figure/LIU-logo.jpg}
	\end{center}
\end{figure}

\end{center}

\end{titlepage}
\clearpage \thispagestyle{empty} ~\clearpage %baksida av försättsblad

%%%%%%%%%%%%%%%%%%% Header %%%%%%%%%%%%%%%%%%%%%%%
\pagestyle{fancy}
\fancyhead[l]{Engineering Applications using Matlab\\Electric Lab-Assistant}
\fancyhead[r]{Dan Helgesson, Albert Skog\\ \& Karl Westerberg}
\fancyfoot[c]{}

%%%%%%%%%%%%%%%%%% Abstract %%%%%%%%%%%%%%%%%%%%%%
\onecolumn
\begin{abstract}
asd
\end{abstract}
\clearpage
%%%%%%%%%%%%%%%%%%% Contents %%%%%%%%%%%%%%%%%%%%%

\tableofcontents
\clearpage
\setcounter{page}{1}
\fancyfoot[c]{\thepage}
\twocolumn


%%%%%%%%%%%%%%%%%% Rapporten %%%%%%%%%%%%%%%%%%%%%
\section{Introduction}

\subsection{Purpose}
\subsection{Methodology}
\subsection{Hypothesis}

\section{Method}
Firstly the 100 users was randomly placed within the circle. The distance from origo is needed for the power received calculationand is calculated by the well known Pythagorean theorem. The power received shown in equ. \eqref{power}  is calculated once for each subcarrier. The only thing that varies in this case is the wavelenght for each center frequency.

\begin{equation}
\label{power}
P_r(d)=\frac{P_rG_tG_r\lambda^2}{(4\pi)^2d^2L}
\end{equation}

A fading is added for each subcarrier for each user. The fading is added because nothing is ideal.%mer text?
To calculate the capacity the Shannon formula in equ. \eqref{shannon} was used. The SNR was computed from the $P_r$ and the noise $N$.

\begin{equation}
\label{shannon}
C=B_wlag_2(1+SNR)
\end{equation}

\section{Results}

\section{Conclusion}


\onecolumn
\appendix
\section{Sample LaTeX output}



\end{document}